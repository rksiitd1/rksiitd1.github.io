\documentclass[11pt,a4paper]{article}
\usepackage[utf8]{inputenc}
\usepackage{amsmath}
\usepackage{hyperref}
\usepackage[margin=1in]{geometry}
\usepackage{fancyhdr}
\usepackage{xcolor}
\usepackage{graphicx}

\definecolor{titlecolor}{RGB}{0, 51, 102}

\pagestyle{fancy}
\fancyhf{}
\rhead{Ratnesh Kumar Sharma}
\lhead{The Computational Limits of Evolution}
\rfoot{Page \thepage}

\title{\textcolor{titlecolor}{The Computational Limits of Evolution: \\ A Challenge to Darwinian Theory Through Universal Scale Analysis}}
\author{
    Ratnesh Kumar Sharma \\
    \texttt{mail.ratneshks@gmail.com} \\
    \url{https://rtnesharma.in/}
}
\date{\today}

\begin{document}

\maketitle

\section*{Introduction}
I have recently conducted an analysis that reveals fundamental challenges to the Darwinian theory of evolution through the lens of computational complexity. This analysis demonstrates that certain aspects of biological evolution may be computationally impossible within our known universe.

\section{The Matrix Analysis}
Consider a simple matrix of switches with 1000 × 1000 rows and columns. Let's calculate the possible number of outcomes:

$$ \text{Total Possible Configurations} = 2^{(1,000 \times 1,000)} = 2^{1,000,000} $$

This number is astronomically larger than $10^{89}$, which represents the total number of atoms in the observable universe. This simple calculation reveals that:

\begin{itemize}
    \item No computer on Earth can calculate all these possibilities
    \item Even if we used the entire universe as a computer, it would be insufficient
\end{itemize}

\section{The Genomic Complexity}
Now, let's consider the human genome:

$$ \text{Possible Genomic Configurations} = 4^{6.4 \times 10^9} $$

This represents the possible permutations with 6.4 billion base pairs. The implications are staggering:
\begin{itemize}
    \item It's computationally impossible to calculate all possibilities
    \item Finding the optimal configuration through random processes becomes implausible
    \item Humans cannot represent the best possible outcome of all possibilities
\end{itemize}

\section{Unexplained Phenomena}
Several critical aspects remain unexplained by current evolutionary theory:

\begin{enumerate}
    \item \textbf{Gender Duality:} There is no comprehensive explanation for the existence of exactly two genders. The origin of the Y chromosome remains a mystery that no one in the world can definitively explain.
    
    \item \textbf{Abiogenesis vs. External Origins:} This computational impossibility suggests we might not have originated from abiogenesis, but potentially from an alien entity.
    
    \item \textbf{Design Complexity:} If we are indeed of alien origin, the question of how we achieved such complexity remains.
    
    \item \textbf{Universal Computational Limits:} Even utilizing every atom in the universe for computation would be insufficient for such optimization.
\end{enumerate}

\section{Theoretical Propositions}
Based on these analyses, I propose:

\begin{enumerate}
    \item This universe alone is insufficient to explain our existence
    
    \item There must exist a "computer" outside our universe that is at least:
    $$ \frac{4^{6.4 \times 10^9}}{10^{89}} $$
    times larger than our entire universe
\end{enumerate}

\section{Quantum Computing Considerations}
Regarding quantum computation:

\begin{itemize}
    \item Future quantum computers may address time complexity issues
    \item However, they cannot solve the fundamental space complexity problem
    \item Space complexity will remain unsolved unless we discover methods to store zettabytes of information in a single atom
    \item It appears fundamentally infeasible to store zettabytes of data within a single hydrogen atom
\end{itemize}

\section{Conclusions}
This analysis reveals fundamental computational limits that challenge our current understanding of evolution and biological complexity. The numbers suggest that our universe alone cannot account for the optimization processes required for current biological complexity. This opens new questions about the nature of our universe and the possibility of larger computational structures beyond our observable cosmos.

\vspace{1cm}
\begin{flushright}
\textit{- Ratnesh Kumar Sharma} \\
\url{https://rtnesharma.in/}
\end{flushright}

\end{document}